\documentclass[20pt,a4paper]{extarticle}
\usepackage[a4paper,margin=6mm]{geometry}
\usepackage{amsmath}
\usepackage{hyperref}
\title{\LaTeX Mathematics Examples}
\author{Anirudh}

\begin{document}
\maketitle
\tableofcontents
\section{Delimiters}
See how the delimiters are of reasonable size in these examples\\
\[
(a+b)\left[1-\frac{b}{a+b}\right]=a,
\]

\[
\sqrt{|xy|}\leq\left|\frac{x+y}{2}\right|,
\]
even when there is no matching delimiter
\[
\int_a^bu\frac{d^2v}{dx^2}\,dx
=\left.u\frac{dv}{dx}\right|_a^b
-\int_a^b\frac{du}{dx}\frac{dv}{dx}\,dx.
\]
\section{Spacing}
Differentials often need a bit of help with their spacing as in
\[
\iint xy^2\,dx\,dy 
=\frac{1}{6}x^2y^3,
\]
whereas vector problems often lead to statements such as
\[
	u=\frac{-y}{x^2+y^2}\,, \quad v=\frac{x}{x^2+y^2}\,, \quad \text{and}\quad w=0
\]
Occasionally one gets horrible line breaks when using a list in mathematics such as listing the first twelve primes \(2,3,5,7,11,13,17,19,23,29,31,37\)\,.
In such cases, perhaps include \verb|\mathcode`\,="213B| inside the inline maths environment so that the list breaks:
\(\mathcode`\,="213B 2,3,5,7,11,13,17,19,23,29,31,37\)\,.
Be discerning about when to do this as the spacing is different
\section{Arrays}

Arrays of mathematics are typeset using one of the matrix environments as 
in
\[
\begin{bmatrix}
1 & x & 0 \\
0 & 1 & -1
\end{bmatrix}\begin{bmatrix}
1  \\
y  \\
1
\end{bmatrix}
=\begin{bmatrix}
1+xy  \\
y-1
\end{bmatrix}\begin{bmatrix}
2\\
3
\end{bmatrix}
\]
Case statements use cases:
\[
|x|=\begin{cases}
x, & \text{if }x>0\,,  \\
-x, & \text{if }x\leq 0\,.
\end{cases}
\]
Many arrays have lots of dots all over the place as in
\[
\begin{matrix}
-2 & 1 & 0 & 0 & \cdots & 0  \\
1 & -2 & 1 & 0 & \cdots & 0  \\
0 & 1 & -2 & 1 & \cdots & 0  \\
0 & 0 & 1 & -2 & \ddots & \vdots \\
\vdots & \vdots & \vdots & \ddots & \ddots & 1  \\
0 & 0 & 0 & \cdots & 1 & -2
\end{matrix}
\]
\section{Equation arrays}
In the flow of a fluid film we may report
\begin{eqnarray}
u_\alpha & = & \epsilon^2 \kappa_{xxx} 
\left( y-\frac{1}{2}y^2 \right),
\label{equ}  \\
v & = & \epsilon^3 \kappa_{xxx} y\,,
\label{eqv}  \\
p & = & \epsilon \kappa_{xx}\,.
\label{eqp}
\end{eqnarray}
Alternatively, the curl of a vector field $(u,v,w)$ may be written 
with only one equation number:
\begin{eqnarray}
\omega_1 & = &
\frac{\partial w}{\partial y}-\frac{\partial v}{\partial z}\,,
\nonumber  \\
\omega_2 & = & 
\frac{\partial u}{\partial z}-\frac{\partial w}{\partial x}\,,
\label{eqcurl}  \\
\omega_3 & = & 
\frac{\partial v}{\partial x}-\frac{\partial u}{\partial y}\,.
\nonumber
\end{eqnarray}
Whereas a derivation may look like
\begin{eqnarray}
	(p\wedge q)\vee(p\wedge\neg q) & = & p\wedge(q\vee\neg q)
	\quad\text{by distributive law}  \\
	& = & p\wedge T \quad\text{by excluded middle}  \\
	& = & p \quad\text{by identity}
\end{eqnarray}

\section{Functions}

Observe that trigonometric and other elementary functions are typeset 
properly, even to the extent of providing a thin space if followed by 
a single letter argument:
\[
\exp(i\theta)=\cos\theta +i\sin\theta\,,\quad
\sinh(\log x)=\frac{1}{2}\left( x-\frac{1}{x} \right).
\]
With sub- and super-scripts placed properly on more complicated 
functions,
\[
\lim_{q\to\infty}\|f(x)\|_q 
=\max_{x}|f(x)|,
\]
and large operators, such as integrals and
\begin{eqnarray}
	e^x & = & \sum_{n=0}^\infty \frac{x^n}{n!}
	\quad\text{where }n!=\prod_{i=1}^n i\,, \nonumber \\
	\overline{U_\alpha} & = & \bigcap_\alpha U_\alpha\,.\nonumber
\end{eqnarray}
In inline mathematics the scripts are correctly placed to the side in 
order to conserve vertical space, as in
\(
1/(1-x)=\sum_{n=0}^\infty x^n.
\)
\section{Accents}

Mathematical accents are performed by a short command with one 
argument, such as
\[
\tilde f(\omega)=\frac{1}{2\pi}
\int_{-\infty}^\infty f(x)e^{-i\omega x}\,dx\,,
\]
\newpage
or
\[
\dot{\vec \omega}=\vec r\times\vec I\,.
\]

\section{Command definition}

\newcommand{\Ai}{\operatorname{Ai}} 
The Airy function, $\Ai(x)$, may be incorrectly defined as this 
integral
\[
\Ai(x)=\int\exp(s^3+isx)\,ds\,.
\]
\end{document}